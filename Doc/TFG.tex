\title{Asistente de composici\'on de m\'usica dodecaf\'onica para OpenMusic}
\author{
        Leopoldo Pla Sempere (lps34 at alu dot ua dot es)\\
        Universidad de Alicante\\
}
\date{\today}

\documentclass[12pt]{article}
\usepackage[utf8]{inputenc} % Para poner acentos y eñes directamente.

\begin{document}
\maketitle

\begin{abstract}
El asistente de composición de música dodecafónica es un conjunto de capas software implementadas sobre varias tecnologías, entre ellas OpenMusic, para crear obras dodecafónicas según unas restricciones impuestas por el usuario. El sistema es capaz de obtener una serie dodecafónica en base a una parcialmente propuesta cumpliendo además ciertas restricciones para luego obtener todas las series derivadas de ella.

Se generan ritmos, saltos de octava y silencios además de poder exportar el resultado a formato MusicXML pudiéndose abrir en cualquier editor de partituras actual. Aquí presentaremos y mostraremos en imágenes el asistente con algunos ejemplos así como de exponer diferentes partes de su desarrollo. También se mostrarán algunos problemas relacionados con el diseño y la programación del programa bajo la plataforma OpenMusic y se sugieren futuras direcciones de desarrollo en esta area.
\end{abstract}

\section{Introduction}
El dodecafonismo es un método de composición musical iniciado por Arnold Schoenberg en 1921-1923 en su obra Suite para piano Op. 25 \cite{styleandidea} después de un silencio donde no publicó nada entre 1914 y 1923 dado que se dió cuenta de que el atonalismo que creó en 1905 (no hay relaciones tonales entre los sonidos) producía limitaciones para realizar estructuras coherentes de larga duración en composiciones y que la única forma de que fuesen extensas era apoyándose en un texto para mantener la cohesión.

\begin{quote}
\em El nuevo método pronto se convirtió en tema de intenso debate y dividió a los compositores de la época: con todo hay que señalar que ejercerá un impacto tremendo en el pensamiento musical del periodo aunque pocos lo adaptaran rigurosamente (entre los que lo hagan están sus alumnos Webern, Berg y Stein).\cite{historiadelamusica}
\end{quote}

Este tipo de composición es un tipo de serialismo (composición basada en series de notas, de ritmos, de dinámicas...) en el que no se busca el peso de notas sobre otras. Es decir, busca lo contrario que la música tonal, que tiene normalmente un tono predominante y unos acordes relacionados (dominante, subdominante, dominante de la dominante...) con mayor peso e importancia sobre otros acordes. Para evitar este énfasis, se crea una serie con las doce notas de la escala cromática  sin repetirse ninguna, de forma que se les da la misma importancia, se crean series derivadas, y se compone solamente con estas construcciones \cite{wiki:twelvetonetechnique}. Esta forma de componer crea, pues, atonalidad.

Para componer con el método dodecafonista primero se ha de obtener, como hemos dicho anteriormente, una matriz donde la primera fila es la serie original (las 12 notas de la escala cromática en un orden) y la primera columna es la serie invertida (serie construida por movimiento contrario de la serie original, es decir, si la primera nota es un Do y la segunda un Re en la serie original, en la invertida será un Do y un Si). Si leemos la fila y columna al contrario (de izquierda a derecha o de abajo a arriba respectivamente) obtenemos la serie retrógrada y la serie retrógrada inversa en cada caso. El resto de elementos de la matriz se puede obtener transportando la primera fila o la primera columna.

\begin{quote}
\em Expuesto de esta manera, el método puede parecer arbitrario. Pero para Schönberg, se trataba de un modo sistemático de llevar a cabo lo que ya estaba haciendo en su música tonal: integrar la armonía y la melodía mediante la composición con un número limitado de conjuntos (aquí, los definidos por los segmentos de la serie), delimitar las frases y subfrases por medio de la saturación cromática (regulada por la aparición de las doce notas en cada exposición de la serie) y apoyarse en la variación en desarrollo. \cite{palisca}
\end{quote}

\paragraph{Outline}
El resto de este artículo está organizado de la siguiente manera. La sección~\ref{overview} explica las plataformas utilizadas para la programación del asistente así como partes de su funcionamiento.
Un ejemplo del funcionamiento del programa se muestra en la sección~\ref{example}.
Finalmente, en la sección~\ref{conclusions} se aportan conclusiones, desarrollos futuros para el proyecto e infomación adicional sobre el trabajo.

\section{Resumen sobre el asistente}\label{overview}
A much longer \LaTeXe{} example was written by Gil.

\section{Ejemplo simple}\label{example}
In this section we describe the results.

\section{Proposiciones e investigaciones posteriores}\label{conclusions}
We worked hard, and achieved very little.

\bibliographystyle{ieeetr}
\bibliography{bibliography}

\end{document}
This is never printed
