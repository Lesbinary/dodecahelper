\documentclass[a4paper,openany,oneside,12pt]{book}
\usepackage[spanish]{babel}
\usepackage[utf8]{inputenc} % Para poner acentos y eñes directamente.

\begin{document}

\chapter*{Resumen}
\addcontentsline{toc}{section}{Resumen} % si queremos que aparezca en el índice
\markboth{RESUMEN}{RESUMEN} % encabezado

El asistente de composición de música dodecafónica es un conjunto de capas software implementadas sobre varias tecnologías, entre ellas OpenMusic y Python, para crear obras dodecafónicas según unas restricciones impuestas por el usuario. El sistema es capaz de obtener una serie dodecafónica en base a una semilla propuesta cumpliendo además ciertas restricciones para luego obtener todas las series derivadas de ella y componer con estas series y otras restricciones.

Se genera una partitura con ritmos, saltos de octava y silencios además de poder exportar el resultado a formato MusicXML para poder seguir editando en cualquier editor de partituras actual. Aquí presentaremos y mostraremos en imágenes el asistente con algunos ejemplos así como de exponer diferentes partes de su desarrollo. También se mostrará la estructura y la programación del programa bajo las diferentes tecnologías utilizadas y se sugieren enfoques para el futuro desarrollo.
\end{document}