\documentclass[a4paper,openany,oneside,12pt]{book}
\usepackage[spanish]{babel}
\usepackage[utf8]{inputenc} % Para poner acentos y eñes directamente.

\begin{document}

\chapter*{Resumen}
\addcontentsline{toc}{section}{Resumen} % si queremos que aparezca en el índice
\markboth{RESUMEN}{RESUMEN} % encabezado

Este trabajo se ha realizado para unir los conocimientos del Grado Profesional de música y el Grado en Ingeniería Informática en un proyecto de composición automática que asista a los compositores a componer obras dodecafónicas, obras interesantes desde el punto de vista matemático y programático.

El dodecafonismo es un método de composición musical basado en una serie de doce notas (las de la escala cromática) donde no existe centro tonal. Esa serie se deriva en una matriz de series que es la utilizada para componer las obras dodecafónicas.

La composición automática es una forma de composición basada en elementos mecánicos, o más frecuentemente electrónicos y digitales, en la toma de decisiones.

Así pues, el asistente de composición de música dodecafónica es un conjunto de capas software implementadas sobre varias tecnologías, entre ellas OpenMusic y Python, para crear obras dodecafónicas según unas restricciones impuestas por el usuario. El sistema es capaz de obtener una serie dodecafónica en base a una semilla propuesta cumpliendo además ciertas restricciones para luego obtener todas las series derivadas de ella y componer con estas series y otras restricciones.

Se genera una partitura con ritmos, saltos de octava y silencios además de poder exportar el resultado a formato MusicXML para poder seguir editando en cualquier editor de partituras actual. Aquí presentaremos y mostraremos en imágenes el asistente con algunos ejemplos así como de exponer diferentes partes de su desarrollo. También se mostrará la estructura y la programación del programa bajo las diferentes tecnologías utilizadas y se sugieren enfoques para el futuro desarrollo.

Solventados los problemas durante el desarrollo incremental de este proyecto, los resultados son satisfactorios y han surgido nuevas ideas para futuros desarrollos derivados, mejoras y expansiones de este proyecto.
\end{document}